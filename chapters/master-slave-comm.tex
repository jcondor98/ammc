\chapter{Master-Slave communication}
\label{ch:master-slave-comm}
The master and slave controllers communicate to each other using the I2C
bus. Naturally, the master controller is responsible for starting communication
sessions, addressing one or all the slaves.

The bitrate is set to 100kbit/s, i.e.\ the maximum speed available for the
original I2C standard\cite{i2c-ref} (excluding fast modes, which were
implemented later).

\subsection{Hardware setup}
The I2C bus itself is a pair of bus rails, named \texttt{SDA} and \texttt{SCL},
representing data and bus clock respectively.

Two $4.7 k\Omega$ resistors connects \texttt{SDA} and \texttt{SCL} to a $5V$
voltage generator and act as \emph{open-drain} resistors.
% TODO: Footnote?

\subsection{Communication frames}
The first byte of the I2C communication frame (excluding, of course, the slave
address) is an 8-bit unsigned integer representing the command sent by the
master to the slave. The trailing data represents an argument, with its size
and shape depending on the type of command issued. Command codes, with their
specific arguments, are shown in table \ref{tab:i2c-commands}.

\begin{table}[bh]
  \begin{tabularx}{\textwidth}{c c c X}
    \toprule
    Command & Code & Arg.\ size (bytes) & Description \\
    \midrule
      \texttt{DC\_MOTOR\_CMD\_GET}   & \texttt{0x00} & - & Get the dc motor speed in rpm \\
      \texttt{DC\_MOTOR\_CMD\_SET}   & \texttt{0x01} & 1 & Set the dc motor target speed \\
      \texttt{DC\_MOTOR\_CMD\_APPLY} & \texttt{0x02} & - & Apply the previously set target speed \\
      \texttt{TWI\_CMD\_ECHO}        & \texttt{0x03} & 1 & Send back the received character to the master (debug)\\
      \texttt{TWI\_CMD\_SET\_ADDR}   & \texttt{0x04} & 1 & Set a different I2C slave address \\
    \bottomrule
  \end{tabularx}
  \caption{Master-to-slave commands}
  \label{tab:i2c-commands}
\end{table}

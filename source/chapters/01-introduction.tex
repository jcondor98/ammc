\chapter{Introduction}
\label{ch:intro}

\section{Motivation}
The use of dc motors is widespread in many application fields. Often, the
offered solutions are monolithic and use of very specific (and sometimes ad-hoc
developed) hardware. The ammc ecosystem focuses on the use of very common,
affordable and easily available hardware, such as AVR microcontrollers.

The software is designed to be modular, extensible and easily hackable by any
average C programmer, so that in the future other people interested in this
project can contribute or adapt ammc to fit their needings and accomplish their
business logic.

\section{Contributions}
The ammc ecosystem offers a flexible, affordable, hackable solution for the
management of multiple dc motors. It is composed of three main components:
\begin{itemize}
  \item A client-side application
  \item A master controller
  \item One or more slave controllers
\end{itemize}

\subsection{Client}
The client-side application is written in C. It makes use of the POSIX standard
library, and in particular the \emph{termios} interface to generic serial
devices. It offers a shell-fashioned terminal user interface.

\subsection{Master}
The master controller is an AVR microcontroller unit. The firmware is written
in C, using the \emph{avr-gcc} compiler and the \emph{avr-libc} standard
library implementation. It dispatches user commands (taken by the client) to
slave controllers.

\subsection{Slave}
The slave controllers are AVR microcontroller units. The firmware is written in
C, using the \emph{avr-gcc} compiler and the \emph{avr-libc} standard library
implementation. Each slave controller manages a single dc motor, taking
commands by the master controller.

\section{Technical approach}

\subsection{Software architecture}
I have developed all the top-level ammc components following the principles I
have learned both from my University courses\cite{fondamenti2-prog} and my
personal insights\cite{clean-architecture}\cite{cpp-ref}.

The client application and the microcontroller firmwares were developed
following the \emph{SOLID} principles, with attention on the \emph{Open-Close
principle}.  I have put particular attention on modularity; many of the
software modules found in all the ammc applications are standalone and will
work out of the box if brought into other projects.

I tried to develop my software with the following policy: the core of the
application should be composed of reusable, standalone modules; the business
logic of the application, which obviously radically changes from project to
project, should be placed in the peripheral modules and should use, when
possible, interfaces for dependency inversion towards the core modules.

For example, in the master controller firmware source code I use function
pointers to generalize the operations to be triggered by client commands.
Naturally, such practices would have been easier and more evident with a
programming language explicitly supporting the object-oriented paradigm (e.g.\
C++).

\subsection{Documentation}
Documentation is paramount in any software project; Damian Conway once said:
"Documentation is a love letter that you write to your future self". I did my
best to write precise and exhaustive documentation for every software piece I
realized.

This thesis is by itself a document dealing with every aspect of the ammc
project. The \texttt{README.md} file in the root directory of the repository
offers a more concise resume.

All the source code (the head files in particular) are exhaustively documented
with \emph{doxygen}, a tool which automatically converts properly formatted
comments found in the code into technical documentation; such resource is
available in the \texttt{gh-pages} git development branch and is hosted using
GitHub Pages.

The client-side application is documented with a Unix man page, which can also
be found in this thesis in section \ref{sec:client-manpage}.

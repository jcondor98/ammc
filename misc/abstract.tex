\begin{abstract}
  Nowadays, men physical work is gradually being taken or eased by technology.
  Many solutions in this area of interest rely on the use of electrical motors
  to accomplish tasks. Furthermore, electrical motors are widely used in many
  retail products, such as toys, domotic devices and so on.

  In this work I propose, as the result of my bachelor degree apprenticeship,
  the \emph{AVR Multi Motor Control} ecosystem as a solution for the management
  of multiple dc motors. It is composed of a \emph{client} application for user
  interaction, a \emph{master} controller and multiple \emph{slave}
  controllers, each handling a single motor. The master controller manages
  slave controllers, reading and manipulating the dc motors' speed with
  granularity.

  In chapter \ref{ch:intro} I give a broad overview on the work done, focusing
  on the structure of the entire project and the approach I had relating to 
  software architecture, coding and documentation writing.

  In chapter \ref{ch:client} I discuss the client-side application, focusing on
  user interaction and software architectural characteristics.

  In chapter \ref{ch:master} I discuss the master controller, focusing on
  hardware setup and firmware internals.

  In chapter \ref{ch:slave} I discuss the slave controller, focusing on
  hardware and the software-defined Proportional-Integral-Derivative controller
  embedded in the firmware.

  In chapter \ref{ch:client-master-comm} I discuss the protocols responsible
  for the communication between the client application and the master
  controller, and how I used the underlying serial communication channel.

  In chapter \ref{ch:master-slave-comm} I explain how I used the I2C bus to
  connect the master controller to multiple slave controllers.
\end{abstract}

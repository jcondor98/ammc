% Bachelor Degree Thesis - Paolo Lucchesi
% !TeX encoding = UTF-8
% !TeX program = pdflatex
% !TeX spellcheck = en_US

\documentclass[binding=0.6cm,Lau]{sapthesis}
\usepackage{microtype}
\usepackage[english]{babel}
\usepackage[utf8]{inputenc}
\usepackage{hyperref}

\hypersetup{pdftitle={AVR Multi Motor Control},pdfauthor={Paolo Lucchesi}}

\title{AVR Multi Motor Control}
\author{Paolo Lucchesi}
\IDnumber{1765134}
\course{Ingegneria Informatica e Automatica}
\courseorganizer{Facoltà di Ingegneria dell'Informazione, Informatica e Statistica}
\AcademicYear{2021/2022}
\copyyear{2021}
\advisor{Prof. Giorgio Grisetti}
\advisor{Drs. Barbara Bazzana}
\reviewerlabel{Reviewer}
\reviewer{Prof. Silvia Bonomi}
\authoremail{lucchesi.1765134@studenti.uniroma1.it}
\website{https://github.com/jcondor98/ammc}


\begin{document}

\frontmatter
\maketitle

\dedication{
Dedicated to my family, my granddad Pietro, my mom Anna Rosa, my dad Marco,
my grandmom Pierina and my sister Valentina, which I thank everyday for
everything I am.\\
To my dearest friends in Pitigliano, with whom I share some of the most
beautiful memories I have.\\
To Nicola, with whom I have shared part of this path; he helped me in a
dark period of my life and he is one of my dearest friends.
}

\begin{abstract}
\end{abstract}

\tableofcontents


\mainmatter

\chapter{Introduction}
\label{ch:intro}


\chapter{Client-side user interaction}
\label{ch:client}
A client program has been realized to manipulate the dc motors directly from the
PC. It can get and set the speed of individual motors, and to apply all the
previously set speeds for all of them at once.\\ A brief list of the
client's features is given below:
\begin{itemize}
  \item Granular handling for getting and setting motors' speed
  \item Modular and extensible software architecture
  \item Terminal User Interface, implemented as a command shell
  \item Support for non-interactive use (i.e.\ scripting)
  \item Communication with master controller using the serial protocol
  \item Compatible with POSIX-compliant environments
\end{itemize}
The client is also documented with a man page, which can be found in section
\ref{sec:client-manpage}.

\section{User Interface}
The end user interacts with the whole ammc ecosystem using a text-based client.
It consists in a shell module, which I had written myself, offering some
\emph{internal commands} (hardcoded in the shell module itself) and is extended
by \emph{external commands} (found in a separated source code entity, and that
can even be compiled in a detached transaction unit).

A particular focus was made on the software architecture: indeed, every
external command can be realized standalone, and it is easy to add new
commands just by altering the \emph{client/source/shell\_commands.c} source
file.

\section{Primitives offered}
The commands that can be used to interface with the ammc ecosystem are the
following:
\begin{description}
  \item[connect <device-path>] Connect to a master controller, given the path
    to the block device representing it.
  \item[get-speed <motor-id>] Get the speed of a dc motor given its id.
    The motor id must be specified as a decimal number.
  \item[set-speed <motor-id>=<speed>] Set the speed of a dc motor given its id.
    The motor id must be specified as a decimal number and the speed must be
    specified in rpm.
  \item[apply] Apply the previously set speed for all the dc motors.
\end{description}

\subsection{Non-interactive mode}
The client shell is capable of running in non-interactive (i.e.\ scripting)
mode with the \emph{-s} option.  If so, it will parse the input from a
specified text file, or from \emph{stdin} if not provided.  A shell launched in
non-interactive mode will not print shell prompts, and exit when end-of-file is
encountered or on command failure.

\section{Serial module}
The client's serial module has been realized using the POSIX \emph{termios}
interface. Unlike the master controller's counterpart, all its code is
reentrant, therefore multiple instances of multiple serial devices can
theoretically exist at the same time.

From the client's perspective, the master controller is seen as a file
descriptor, and the end user just have to specify the path of the block device
file representing the serial communication channel (e.g. \emph{/dev/ttyACM0})
using the \emph{connect} command.

\section{Specification}
\subsection{Software modules}
\subsection{Modules dependency graph}

\section{Man page}
\label{sec:client-manpage}


\chapter{Master controller}
\label{ch:master}
The master controller handles all the slave controllers, dispaching arbitrary
commands to them using the I2C protocol. It also communicates directly with the
client application via serial port.

\section{Hardware setup}
The master controller itself is an AVR \emph{ATMega2560} microcontroller unit.
This particular MCU has some convenient features, such as:
\begin{itemize}
  \item I2C dedicated hardware subsystem
  \item Serial-over-USB bridge
  \item Relatively powerful specifications for future feature adding
  \item Plenty of timers and outgoing power pins
\end{itemize}

\section{I2C setup}
\section{Power management}
\section{Specification}
\subsection{Software modules}
\subsection{Modules dependency graph}
\subsection{Circuit schematics}
\subsection{Wiring}



\chapter{Slave controllers}
\label{ch:slave}
\section{Hardware setup}
\section{I2C setup}
\section{Power management}
\section{Specification}
\subsection{Software modules}
\subsection{Modules dependency graph}
\subsection{Circuit schematics}
\subsection{Wiring}

\chapter{Client-Master communication}
\label{ch:client-master-comm}

\chapter{Master-Slave communication}
\label{ch:master-slave-comm}

\chapter{Conclusions}
\label{ch:conclusions}


\backmatter
\cleardoublepage
\phantomsection % Give this command only if hyperref is loaded
\addcontentsline{toc}{chapter}{\bibname}
% Here put the code for the bibliography. You can use BibTeX or
% the BibLaTeX package or the simple environment thebibliography.
\end{document}

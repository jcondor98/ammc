\chapter{Client-Master communication}
\label{ch:client-master-comm}
The client application and the master controller communicate over a protocol
built on top of the serial-over-usb layer. Such protocol is completely binary
and packet-based. Each packet has a variable length (with a total maximum size
of 36 bytes) and its integrity is checked with a trailing CRC-8 checksum.

The protocol comes with a simple handshaking mechanism, in order to synchronize
the packet IDs between endpoints and as a shallow proof of correct
functionality of the communication layers.

\section{Serial layer}
Data exchange between client and master relies on the serial protocol. For the
client application, the \emph{termios} library is used, while for the master
controller the (hardware) serial subsystem offered by the AVR
microcontroller\cite{at2560-ref} is used.

The serial communication is set up as follows:
\begin{itemize}
  \item Baud rate of 115200 baud, double speed transmission
  \item 8 bit frame width
  \item No start bits, 1 stop bit (i.e.\ 8N1)
  \item No embedded parity bit
\end{itemize}

For the master controller, both transmission and reception are
interrupt-driven, so the communication main routine is only called when data is
actually available.

\section{Packet headers}
Each packet's metadata is contained in a fixed size header, composed as
describe in table \ref{tab:packet-header}.

\begin{table}[bh]
  \begin{tabularx}{\textwidth}{c c X}
    \toprule
    Field & Size (bits) & Description \\
    \midrule
    id       & $8$ & Packet ID \\
    type     & $8$ & Packet type \\
    selector & $8$ & Selector for DC motors. Also used to store error codes in NAK packets \\
    size     & $8$ & Total packet size, including header and checksum \\
    \bottomrule
  \end{tabularx}
  \caption{Packet header fields}
  \label{tab:packet-header}
\end{table}

\subsection{Identifiers}
The \emph{id} field stores the packet identifier, which is incremental and can
be repeated in a single communication session. When the handshake is performed,
or when any endpoint raises an error (issuing a \emph{NAK} packet), the id is
reset to zero for both sides. Each \emph{ACK} and \emph{NAK} packet is
generated with the same id of the referred packet (see
\ref{client-master-comm-ackerr} for further informations).

\subsection{Packet types}
The \emph{type} field stores the packet type. An exhaustive list of packet
types is given in table \ref{tab:packet-types}

\begin{table}[bh]
  \begin{tabularx}{\textwidth}{c c X}
    \toprule
    Type & Code & Description \\
    \midrule
    \texttt{NULL}       & \texttt{0x00} & Reserved, never use \\
    \texttt{HND}        & \texttt{0x01} & Handshake \\
    \texttt{ACK}        & \texttt{0x02} & Acknowledgement \\
    \texttt{NAK}        & \texttt{0x03} & Communication error \\
    \texttt{ECHO}       & \texttt{0x04} & Echo between client and master (debug only)\\
    \texttt{TWI\_ECHO}  & \texttt{0x05} & Echo a single char to the first slave via TWI (debug only)\\
    \texttt{GET\_SPEED} & \texttt{0x06} & Get the current speed for a dc motor \\
    \texttt{SET\_SPEED} & \texttt{0x07} & Set (and apply) the speed for a dc motor \\
    \texttt{APPLY}      & \texttt{0x08} & Tell all the slaves to apply the previously set speeds \\
    \texttt{DAT}        & \texttt{0x09} & Primarily used for responses from the AVR device \\
    \texttt{SET\_ADDR}  & \texttt{0x0A} & Change the I2C address of a slave \\
    \texttt{LIMIT}      & \texttt{0x0B} & Used for sanity checks - Must have highest value \\
    \bottomrule
  \end{tabularx}
  \caption{Exhaustive list of packet types}
  \label{tab:packet-types}
\end{table}

\subsection{Motor selector}
The \emph{selector} field is used to store the identifier for the dc motor to
manipulate. This field is used in \texttt{GET\_SPEED} and \texttt{SET\_SPEED} packets.

For \emph{NAK} packets, instead, the \emph{selector} field is used to store
error codes (found in table \ref{tab:packet-error-codes}).

For all the other packet types, the \emph{selector} field is ignored.

\section{Acknowledgements and errors}
\label{client-master-comm-ackerr}
A communication endpoint must wait for an acknowledgement message from the
counterpart once it sent a packet in order to send a new one. ACK and NAK
packets do not bring any data.

When a packet arrives, it is checked for integrity and sanity. If it is sane,
then an ACK packet is sent; if not, then a NAK packet is sent.
ACK and NAK packets are simply discarded if corrupted in some way.

A NAK packet uses the \emph{selector} field to send to the other endpoint the
error code describing what happened on its side. Error codes are listed in
table \ref{tab:packet-error-codes}.

\begin{table}[bh]
  \begin{tabularx}{\textwidth}{c c X}
    \toprule
    Error & Code & Description \\
    \midrule
    \texttt{SUCCESS}             & \texttt{0x00} & No errors encountered \\
    \texttt{ID\_MISMATCH}        & \texttt{0x01} & Id of received packet is not consistent \\
    \texttt{CORRUPTED\_CHECKSUM} & \texttt{0x02} & Checksum mismatch, received packet is corrupted \\
    \texttt{WRONG\_TYPE}         & \texttt{0x03} & Received packet has invalid type \\
    \texttt{TOO\_BIG}            & \texttt{0x04} & Received packet has invalid size (too big) \\
    \bottomrule
  \end{tabularx}
  \caption{Exhaustive list of packet types}
  \label{tab:packet-error-codes}
\end{table}

\subsection*{NAME}
\textbf{ammc} — AVR Temperature Monitor client

\subsection*{SYNOPSIS}
\textbf{ammc} [\textbf{-s} \emph{[script]}] [\textbf{-c} \emph{device-file}] \\
\textbf{ammc} -h

\subsection*{DESCRIPTION}
\textbf{ammc} is a multi-motor control device realized with an AVR microcontroller;
this program is an ad-hoc, tui client to interface it.  
The ammc device is designed to work with DC motors with an embedded or external
encoder.

\subsubsection*{Options}
\begin{description}
\item[-c \emph{device-file}]
\ \ Connect to the device identified by \emph{device-file} (e.g. /dev/ttyACM0)

\item[-s \emph{[script]}]
\ \ Execute in script mode (do not print prompt, exit on error etc...), being
\ \ \emph{script} a file containing a command each line. If \emph{script} is not given,
\ \ standard input is used

\item[-h]
\ \ Display a help message and exit
\end{description}

\subsubsection*{Commands}
\begin{description}
  \item[\textbf{help} [\emph{command}\textrm{]}]
  \ \ Show help, also for a specific command if an argument is given

  \item[\textbf{connect} <\emph{device\_path}>]
  \ \ Connect to an ammc Master MCU given its device file (usually under /dev)

  \item[\textbf{disconnect}]
  \ \ Close an existing connection - Has no effect on the device

  \item[\textbf{dev-echo} <\emph{arg}> [\emph{arg2} \emph{arg3} \emph{...}\textrm{]}]
  \ \ Send a string to the device, which should send it back

  \item[\textbf{get-speed} <\emph{motor-id}>]
  \ \ Get the speed of a DC motor in RPM

  \item[\textbf{set-speed} <\emph{motor-id}>=<\emph{value}>]
  \ \ Set the speed of a DC motor in RPM
\end{description}

\subsection*{AUTHOR}
Paolo Lucchesi <paololucchesi@protonmail.com> \\
https://www.github.com/jcondor98/ammc
